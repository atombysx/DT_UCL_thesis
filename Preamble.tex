% I may change the way this is done in a future version, 
%  but given that some people needed it, if you need a different degree title 
%  (e.g. Master of Science, Master in Science, Master of Arts, etc)
%  uncomment the following 3 lines and set as appropriate (this *has* to be before \maketitle)
% \makeatletter
% \renewcommand {\@degree@string} {Master of Things}
% \makeatother

\title{When Vision Knows Where: Visual and Spatial Representations in Dynamically Changing Spatial Contexts}
\author{Diao Tong}
\department{Department of Experimental Psychology, Institute of Behavioural Neuroscience}

\maketitle
\makedeclaration

\begin{abstract} % 300 word limit
Vision plays a key role in spatial navigation. Traditionally, visual processing has been viewed as feedforward, providing visual input to navigation-related areas like the hippocampus. However, recent findings challenge this view, showing that spatial representation of an environment can emerge in early visual areas including primary visual cortex (V1). Nevertheless, it remains unclear if spatial signals in the visual system differ by spatial context. 

 This project investigates if mouse visual and spatial areas can form distinct spatial representations of two virtual reality (VR) environments and how these representations reshape and interact with each other when the environments dynamically change. 

 To test this, I trained mice to distinguish identical visual landmarks to obtain water rewards in one of two VR environments containing multiple sets of identical visual landmarks. Following training I used dual Neuropixel probes to record simultaneously from hundreds of V1 and medial entorhinal cortex (MEC) neurons during navigation of both environments.

 Mice differentiated visual landmarks at different positions to actively obtain water rewards and were able to adapt behaviour in a novel environment having different reward positions. 

 In V1, activity at both individual neuron and population-level can separate the identical landmarks across positions and contexts. The positional modulations do not simply transfer from the familiar environment to the novel environment. In MEC, neurons have diverse functional tunings to spatial position, reward, speed and grid-like period firing. MEC population codes have separate representations of the two environments and grid cell-like patterns are found in low-dimensional space. Furthermore, trial-to-trial behavioural variability dynamically influenced neural responses in both V1 and MEC. 

 Overall, through behavioural manipulations, V1 and MEC neuronal populations can rapidly form and reshape spatial representations of dynamically changing environments. Further work investigating trial-to-trial variability will provide further insights into how the brain integrates contextual information to perform navigation tasks.

\end{abstract}

\begin{impactstatement}

	UCL theses now have to include an impact statement. \textit{(I think for REF reasons?)} The following text is the description from the guide linked from the formatting and submission website of what that involves. (Link to the guide: {\scriptsize \url{http://www.grad.ucl.ac.uk/essinfo/docs/Impact-Statement-Guidance-Notes-for-Research-Students-and-Supervisors.pdf}})

\begin{quote}
The statement should describe, in no more than 500 words, how the expertise, knowledge, analysis,
discovery or insight presented in your thesis could be put to a beneficial use. Consider benefits both
inside and outside academia and the ways in which these benefits could be brought about.

The benefits inside academia could be to the discipline and future scholarship, research methods or
methodology, the curriculum; they might be within your research area and potentially within other
research areas.

The benefits outside academia could occur to commercial activity, social enterprise, professional
practice, clinical use, public health, public policy design, public service delivery, laws, public
discourse, culture, the quality of the environment or quality of life.

The impact could occur locally, regionally, nationally or internationally, to individuals, communities or
organisations and could be immediate or occur incrementally, in the context of a broader field of
research, over many years, decades or longer.

Impact could be brought about through disseminating outputs (either in scholarly journals or
elsewhere such as specialist or mainstream media), education, public engagement, translational
research, commercial and social enterprise activity, engaging with public policy makers and public
service delivery practitioners, influencing ministers, collaborating with academics and non-academics
etc.

Further information including a searchable list of hundreds of examples of UCL impact outside of
academia please see \url{https://www.ucl.ac.uk/impact/}. For thousands more examples, please see
\url{http://results.ref.ac.uk/Results/SelectUoa}.
\end{quote}
\end{impactstatement}

\begin{acknowledgements}
Acknowledge all the things!
\end{acknowledgements}

\setcounter{tocdepth}{2} 
% Setting this higher means you get contents entries for
%  more minor section headers.

\tableofcontents
\listoffigures
\listoftables

